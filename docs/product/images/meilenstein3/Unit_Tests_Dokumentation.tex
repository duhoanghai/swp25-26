\documentclass{article}
\usepackage[utf8]{inputenc}
\usepackage[T1]{fontenc}
\usepackage[ngerman]{babel}
\usepackage{amsmath}
\usepackage{graphicx}
\usepackage{hyperref}

\begin{document}

\section{Dokumentation der Testdurchführung}

\subsection{Ziel}
Die Unit-Tests sollen zeigen, ob unsere Datenklassen richtig funktionieren und ob der
\texttt{QuestionTypeAdapter} die JSON-Daten richtig einliest.

\subsection{Testumgebung}
Alle Tests wurden in Android Studio unter \texttt{src/test/java} ausgeführt.
Die Tests laufen ohne Handy, nur auf der JVM. 

\subsection{Durchführung}

\subsubsection{AnswerTest}
Hier wurde geprüft, ob die \texttt{Answer}-Klasse die Werte richtig speichert.
Das hat funktioniert.

\subsubsection{QuestionTest}
Es wurde getestet, ob die verschiedenen Frageklassen
(z.\,B.\ Slider, Yes/No, Multiple Choice) die Daten korrekt übernehmen.
Auch das funktioniert.

\subsubsection{QuestionTypeAdapterTest}
Hier wurde geprüft, ob der Adapter aus dem \texttt{type}-Feld im JSON
die richtige Frageart macht.  
Beispiele:
\begin{itemize}
    \item \texttt{"QuestionSlider"} wird zu \texttt{QuestionSlider}
    \item \texttt{"QuestionTime"} wird zu \texttt{QuestionTime}
\end{itemize}

Außerdem wurden zwei Fehler getestet:
\begin{itemize}
    \item wenn \texttt{type} im JSON fehlt
    \item wenn ein falscher \texttt{type} drinsteht
\end{itemize}
In beiden Fällen wurde richtig eine Fehlermeldung ausgelöst.

\subsection{Ergebnis}
Alle Tests waren erfolgreich.  
Die Daten werden richtig eingelesen und falsches JSON wird erkannt.

\subsection{Fazit}
Die wichtigsten Funktionen für den Fragebogen arbeiten wie erwartet.
Damit ist die Grundlage für die weitere Entwicklung gelegt.



\end{document}