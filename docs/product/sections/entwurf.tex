\section{Architekturbeschreibung und Beschreibung Integrationstests}

\subsection{Bausteinschicht}
\subsubsection{Komponentendiagramm}
\begin{figure}[h]
	\centering
	\includegraphics[width=0.5\textwidth]{images/meilenstein2/komponentendiagramme.pdf} 
	\caption{Komponentendiagramm}
	\label{fig:komponentendiagramm}
\end{figure}

\subsubsection{Klassendiagramm}
\begin{figure}[h]
	\centering
	\includegraphics[width=1\textwidth]{images/meilenstein2/klassendiagramme.pdf} 
	\caption{Klassendiagramm}
	\label{fig:klassendiagramm}
\end{figure}

\newpage

\subsection{Laufzeitsicht}
\subsubsection{Anmeldung}
\begin{figure}[h]
	\centering
	\includegraphics[width=0.7\textwidth]{images/meilenstein2/Sequenz_Anmeldung.drawio.png} 
	\caption{Anmeldung 1}
	\label{fig:anmeldung1}
\end{figure}

\begin{figure}[h]
	\centering
	\includegraphics[width=0.7\textwidth]{images/meilenstein2/Sequenz_Anmeldung2.drawio.png} 
	\caption{Anmeldung 2}
	\label{fig:anmeldung2}
\end{figure}

\newpage 

\begin{figure}[h]
	\centering
	\includegraphics[width=1\textwidth]{images/meilenstein2/Aktivitaet_Anmeldung.drawio.png} 
	\caption{Anmeldung 3}
	\label{fig:anmeldung3}
\end{figure}

\subsubsection{Einwilligung erteilen}
\begin{figure}[h!]
	\centering
	\includegraphics[width=1\textwidth]{images/meilenstein2/sqd_Einwilligung_erteilen_angepasst.png} 
	\caption{Einwilligung erteilen}
	\label{fig:einwilligung_erteilen}
\end{figure}

\newpage

\subsubsection{Einstellungen verwalten}
\begin{figure}[h]
	\centering
	\includegraphics[width=1\textwidth]{images/meilenstein2/Einstellungen_verwalten_Sprache .drawio.png} 
	\caption{Einstellungen verwalten Sprache}
	\label{fig:einstellungen_verwalten_sprache}
\end{figure}

\begin{figure}[h]
	\centering
	\includegraphics[width=1\textwidth]{images/meilenstein2/Einstellungen_verwalten_Zeit.png} 
	\caption{Einstellungen verwalten Zeit}
	\label{fig:einstellungen_verwalten_zeit}
\end{figure}

\newpage

\subsubsection{Schlafprotokoll ausfüllen}
\begin{figure}[h]
	\centering
	\includegraphics[width=0.8\textwidth]{images/meilenstein2/Protokoll_ausfullen.drawio.png} 
	\caption{Schlafprotokoll ausfüllen}
	\label{fig:schlafprotokoll_ausfüllen}
\end{figure}

\newpage

\subsubsection{Bericht ansehen}
\begin{figure}[h!]
	\centering
	\includegraphics[width=0.8\textwidth]{images/meilenstein2/Bericht_ansehen_Patient.drawio.png} 
	\caption{Bericht ansehen Patient:in}
	\label{fig:bericht_ansehen_patient}
\end{figure}

\subsubsection{Bericht ansehen}
\begin{figure}[h!]
	\centering
	\includegraphics[width=0.8\textwidth]{images/meilenstein2/Bericht_ansehen_Arzt.drawio.png} 
	\caption{Bericht ansehen Ärzt:in}
	\label{fig:bericht_ansehen_arzt}
\end{figure}

\newpage

\subsection{Verteilungssicht}
\subsubsection{Deploymentdiagramm}
\begin{figure}[h]
	\centering
	\includegraphics[width=0.8\textwidth]{images/meilenstein2/deploymentdiagramm.pdf} 
	\caption{Deploymentdiagramm}
	\label{fig:deploymentdiagramm}
\end{figure}

\newpage

\subsection{Integrationstests}

\subsubsection*{1. Schlafprotokoll ausfüllen}

\textbf{Test 1.1: Erfolgreich neues Schlafprotokoll speichern} \\
\begin{tabular}{|p{4cm}|p{10cm}|}
    \hline
    Ziel & Prüfen, ob ein neuer Protokolleintrag korrekt von GUI \\
    \hline
    Vorbedingungen & Patient ist angemeldet \\
    \hline
    Eingaben (GUI) & Einschlafzeit: 22:30, Aufwachzeit: 07:10, Befunden: 4 und Datum wird automatisch eingetragen \\
    \hline
    Testschritte & 
	1. Patient öffnet Protokollformular, GUI ruft \textit{ausfüllenFragebogen()} auf
	2. GUI sendet Daten an Backend via \textit{neuenEintrag()} 
	3. Backend validiert Daten 
	4. Backend ruft DB-Methode \textit{insert(Schlafprotokoll)} auf \\
    \hline
    Erwartete Backend-DB Interaktionen & \textit{insert()} wird einmal mit vollständigem Datensatz ausgeführt \\
    \hline
    Erwartete Ausgabe (GUI) & „Eintrag erfolgreich erstellt" \\
    \hline
    Fehler & Keine \\
    \hline
\end{tabular}

\vspace{0.5cm}

\textbf{Test 1.2: Ungültige Schlafzeiten} \\
\begin{tabular}{|p{4cm}|p{10cm}|}
    \hline
    Ziel & Überprüfung von eingetragenen Daten \\
    \hline
    Eingabe & Einschlafzeit: 23:00, Aufwachzeit: 22:00 (am selben Tag) \\
    \hline
    Erwartetes Ergebnis & Backend erkennt den Fehler, sendet eine Fehlermeldung an GUI wie z.B. „ungültige Zeiten" und \textit{insert()} wird nicht ausgeführt. \\
    \hline
\end{tabular}

\vspace{1cm}

\subsubsection*{2. Einwilligung erteilen}

\textbf{Test 2.1: Einwilligung vorhanden} \\
\begin{tabular}{|p{4cm}|p{10cm}|}
    \hline
    Ziel & Prüfen, dass Einwilligung korrekt gespeichert wird \\
    \hline
    Vorbedingung & Patient gibt Zugangscode eine \\
    \hline
    Schritte & 
    1. GUI → Backend: \textit{informationErfragen()} 
    2. Backend → DB: \textit{getConsentFormular(patientID)} 
    3. Patient klickt „Einverstanden" → GUI → Backend: \textit{einwilligungSpeichern()} 
    4. Backend → DB: \textit{save(patientID)} \\
    \hline
    Erwartet & DB speichert \textit{einverständnis=true}, GUI zeigt „Zugang freigeschaltet" \\
    \hline
\end{tabular}

\vspace{0.5cm}

\textbf{Test 2.2: Einwilligung verweigert} \\
\begin{tabular}{|p{4cm}|p{10cm}|}
    \hline
    Ziel & Verhindern, dass Minderjährige ohne Zustimmung starten \\
    \hline
    Schritte & Gleich wie oben, aber Patient „Ablehnen" \\
    \hline
    Erwartet & DB speichert \textit{false}, GUI zeigt „Zugang nicht möglich" \\
    \hline
\end{tabular}

\vspace{1cm}

\subsubsection*{3. Bericht ansehen}

\textbf{Test 3.1: PDF wird Korrekt generiert} \\
\begin{tabular}{|p{4cm}|p{10cm}|}
    \hline
    Ziel & Prüfen, ob Backend die Richtigen Daten holt und PDF generiert. \\
    \hline
    Schritte & 
    1. Arzt wählt Patient 
    2. GUI → Backend: \textit{berichtErstellen(patientID)} 
    3. Backend → DB: \textit{getReport(patientID)} 
    4. Backend generiert PDF via \textit{generatePDF()} 
    5. PDF an GUI senden \\
    \hline
    Erwartet & PDF enthält aller Einträge der Schlafprotokoll einer Patient \\
    \hline
\end{tabular}

\vspace{0.5cm}

\textbf{Test 3.2: Keine Einträge vorhanden} \\
\begin{tabular}{|p{4cm}|p{10cm}|}
    \hline
    Kategorie & Beschreibung \\
    \hline
    Eingabe & PatientID existiert, aber keine Protokolle \\
    \hline
    Erwartet & Backend liefert eine PDF ohne Datensätze \\
    \hline
\end{tabular}

\vspace{1cm}

\subsubsection*{4. Anmelden}

\textbf{Test 4.1: Login möglich} \\
\begin{tabular}{|p{4cm}|p{10cm}|}
    \hline
    Ziel & Prüfen, dass Registrierung funktioniert \\
    \hline
    Schritte & 
    1. GUI → Backend: \textit{validateCode(code)} 
    2. Backend → DB: \textit{checkCode()} 
    3. GUI → Backend: \textit{setPassword()} 
    4. Backend → DB: \textit{changePassword()} \\
    \hline
    Erwartet & Passwort gehasht gespeichert, Login möglich \\
    \hline
\end{tabular}

\vspace{0.5cm}

\textbf{Test 4.2: Kein Login möglich} \\
\begin{tabular}{|p{4cm}|p{10cm}|}
    \hline
    Kategorie & Beschreibung \\
    \hline
    Eingabe & Ungültiger Code als Passwort eingeben \\
    \hline
    Erwartet & Fehlermeldung: „Ungültiger Code" kein Passwort darf gespeichert werden \\
    \hline
\end{tabular}

\vspace{1cm}

\subsubsection*{5. Einstellung verwalten}

\textbf{Test 5.1: Sprache erfolgreich ändern} \\
\begin{tabular}{|p{4cm}|p{10cm}|}
    \hline
    Schritte & 
    GUI → Backend: \textit{einstellungAnpassen("EN")} 
    Backend → DB: \textit{updateUserSettings()} 
    Backend lädt \textit{loadLanguageResources()} \\
    \hline
    Erwartet & GUI wird vollständig auf Englisch aktualisiertwerden \\
    \hline
\end{tabular}

\vspace{0.5cm}

\textbf{Test 5.2: SprachPaket fehlt} \\
\begin{tabular}{|p{4cm}|p{10cm}|}
    \hline
    Schritte & Gleich wie oben \\
    \hline
    Erwartet & Backend-Fehler → GUI: „Sprache nicht verfügbar" \\
    \hline
\end{tabular}

\vspace{0.5cm}

\textbf{Test 5.3: Erinnerung erfolgreich ändern} \\
\begin{tabular}{|p{4cm}|p{10cm}|}
    \hline
    Schritte & 
    GUI → Backend: \textit{einstellungAnpassen(benachrichtigungsZeit)} 
    Backend → DB: \textit{updateUserSettings()} 
    Backend setzt Timer \\
    \hline
    Erwartet & Neue Zeit erscheint im GUI \\
    \hline
\end{tabular}

\vspace{0.5cm}

\textbf{Test 5.4: Ungültige Zeit auswählen} \\
\begin{tabular}{|p{4cm}|p{10cm}|}
    \hline
    Eingabe & Uhrzeit: 99:99 \\
    \hline
    Erwartet & Fehlermeldung: „Ungültiger Code" kein Passwort darf gespeichert werden \\
    \hline
\end{tabular}

